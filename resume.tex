%!TEX TS-program = xelatex
%!TEX encoding = UTF-8 Unicode
% Awesome CV LaTeX Template
%
% This template has been downloaded from:
% https://github.com/posquit0/Awesome-CV
%
% Author:
% Claud D. Park <posquit0.bj@gmail.com>
% http://www.posquit0.com
%
% Template license:
% CC BY-SA 4.0 (https://creativecommons.org/licenses/by-sa/4.0/)
%


%%%%%%%%%%%%%%%%%%%%%%%%%%%%%%%%%%%%%%
%     Configuration
%%%%%%%%%%%%%%%%%%%%%%%%%%%%%%%%%%%%%%
%%% Themes: Awesome-CV
\documentclass[]{awesome-cv}
\usepackage{textcomp}
%%% Override a directory location for fonts(default: 'fonts/')
\fontdir[fonts/]

%%% Configure a directory location for sections
\newcommand*{\sectiondir}{resume/}

%%% Override color
% Awesome Colors: awesome-emerald, awesome-skyblue, awesome-red, awesome-pink, awesome-orange
%                 awesome-nephritis, awesome-concrete, awesome-darknight
%% Color for highlight
% Define your custom color if you don't like awesome colors
\colorlet{awesome}{awesome-red}
%\definecolor{awesome}{HTML}{CA63A8}
%% Colors for text
%\definecolor{darktext}{HTML}{414141}
%\definecolor{text}{HTML}{414141}
%\definecolor{graytext}{HTML}{414141}
%\definecolor{lighttext}{HTML}{414141}

%%% Override a separator for social informations in header(default: ' | ')
%\headersocialsep[\quad\textbar\quad]
    \begin{document}
    
%%%%%%%%%%%%%%%%%%%%%%%%%%%%%%%%%%%%%%
%     Profile
%%%%%%%%%%%%%%%%%%%%%%%%%%%%%%%%%%%%%%
\begin{center}
	\headerfirstnamestyle{Daniel} \headerlastnamestyle{Hernandez} \\
	\vspace{2mm}
	{\hspace{0.8cm}\faEnvelope\ daniher02@gmail.com}  |  {\faMobile\ (+34) 624382949}  |  {\faMapMarker\ Málaga, España} 
	\newline {\faLink\ \href{https://www.linkedin.com/in/dhernandez-dev/}{https://www.linkedin.com/in/dhernandez-dev/}}
\end{center}

\cvsection{Habilidades}
\begin{cventries}
	\cventry
	{}
	{\def\arraystretch{1.15}{\begin{tabular}{ l l }
		Languages:  & {\skill{ Python, R, PHP, JavaScript, TypeScript, SQL, HTML, CSS}} \\
		Frameworks:  & {\skill{ Django, Fast Api, React, Astro.}} \\
		Databases:  & {\skill{ MySQL, PostgreSQL, SQLite, Redis.}} \\
		Technologies / Tools: \hspace{0.05cm} & {\skill{ Docker, Caprover, pipenv, Celery, npm, Git, GitHub Actions, Api Rest.}} \\
		Practices:  & {\skill{ Agile, SOLID Principles, Code Reviews.}} \\
		\end{tabular}}}
	{}
	{}
	{}
\end{cventries}
\vspace{-7mm}
%%%%%%%%%%%%%%%%%%%%%%%%%%%%%%%%%%%%%%
%     Experience
%%%%%%%%%%%%%%%%%%%%%%%%%%%%%%%%%%%%%%
\cvsection{Expeciencia}
\begin{cventries}
	\cventry
	{Desarrollador Fullastack - Freelance}
	{Club Universidad de Chile - Departamento D+I}
	{Santiago, Chile}
	{Septiempre 2024 – Actualmente}
	{\begin{cvitems}
		\vspace{0.5mm}
		\item {Construir una plataforma web de centralización y visualización de datos de las distintas áreas para el departamento de desarrollo e innovación. Desarrollado con Django y Javascript, desplegado en Azure con Docker y Caprover.}
		\item {Automatizar la carga y migración de datos desde Excel a Bases de datos en Postgres, reduciendo el tiempo del proceso de traspaso de datos en un 90\%.}
		\end{cvitems}}

	\cventry
	{Desarrollador Fullastack}
	{Agencia de Marketing Digital IdeasPro}
	{Talca Chile}
	{Marzo 2022 – Febrero 2025}
	{\begin{cvitems}
		\vspace{0.5mm}
		\item {Liderar un equipo de 4 personas en los desarrollos de los distintos proyectos de software para los clientes de la empresa.}
		\item {Diseñar y Desarrollar una plataforma para la gestión de datos administrativos y clinicos para empresas de Radiografías Dentales. Agilizando el proceso de atención y gestion de pacientes en un 50\%.}
		\item {Diseñar, Desarrollar y Desplegar una plataforma de gestión de ordenes de trabajos de impresión 3D para una empresa de impresiones 3D dentales.}
		\item {Desarrollar Páginas web para distintos clientes de la empresa. Utilizando Astro, React y desplegadas en Vercel.}
		\end{cvitems}}

	\cventry
	{Desarrollador Fullastack - Freelance}
	{Servicios Postales}
	{Talca, Chile}
	{Enero 2022 – Agosto 2022}
	{\begin{cvitems}
		\vspace{0.5mm}
		\item {Desarrollar una plataforma web para la gestión de envíos de paquetes y cartas. Desarrollado con PHP, Javascript y MySQL.}
		\end{cvitems}}
\end{cventries}

\cvsection{Proyectos}
\begin{cventries}
	\vspace{-3mm}
	\cventry
	{}
	{Resumenes Estadísticos \vspace{-5mm}}
	{Python, Pandas, Matplotlib, WebScraping \vspace{-5mm}}
	{}
	{\begin{cvsectionnormaltext}
		\item {Automaticé el proceso de visualizar un resumen estadístico obtenido desde la data de Sofascore. 
		\newline \faLink\ \href{https://github.com/Daniher01/resumenes_estadisticos}{https://github.com/Daniher01/resumenes\_estadisticos}}
	\end{cvsectionnormaltext}}

	\vspace{-3mm}
	\cventry
	{}
	{Scouting de Jugadores \vspace{-5mm}}
	{R, Shiny App, StatsBomb \vspace{-5mm}}
	{}
	{\begin{cvsectionnormaltext}
		\item {Realicé una plataforma con datos y métricas avanzadas de statsbomb donde permite buscar jugadores y ver similitudes entre ellos.
		\newline \faLink\ \href{https://dhernandezm.shinyapps.io/scouting_app/}{https://dhernandezm.shinyapps.io/scouting\_app/}}
	\end{cvsectionnormaltext}}
	
	\vspace{-3mm}
	\cventry
	{}
	{Gestión de Filas \vspace{-5mm}}
	{Django, Javascript, Docker, PostgreSQL, WebSocket \vspace{-5mm}}
	{}
	{\begin{cvsectionnormaltext}
		\item {Desarrollé una plataforma para la gestión de filas de pacientes. Siendo principalente para aprender a implementar websockets par la comunicación a tiempo real entre clientes. 
		\newline \faLink\ \href{https://github.com/Daniher01/gestion_filas}{https://github.com/Daniher01/gestion\_filas}}
	\end{cvsectionnormaltext}}
	
	\vspace{-5mm}
	
\end{cventries}

%%%%%%%%%%%%%%%%%%%%%%%%%%%%%%%%%%%%%%
%     Education
%%%%%%%%%%%%%%%%%%%%%%%%%%%%%%%%%%%%%%
\vspace{8mm}
\cvsection{Educación}
\begin{cventries}
	\vspace{-3mm}
	\cventry
	{}
	{CFT San Agustín \vspace{-5mm}}
	{Talca, Chile \vspace{-5mm}}
	{2020- 2022 \vspace{-5mm}}
	{\begin{cvsectionnormaltext} 
		\item{Analista Programador}
	\end{cvsectionnormaltext}}
\end{cventries}

\end{document}