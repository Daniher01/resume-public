%!TEX TS-program = xelatex
%!TEX encoding = UTF-8 Unicode
% Awesome CV LaTeX Template
%
% This template has been downloaded from:
% https://github.com/posquit0/Awesome-CV
%
% Author:
% Claud D. Park <posquit0.bj@gmail.com>
% http://www.posquit0.com
%
% Template license:
% CC BY-SA 4.0 (https://creativecommons.org/licenses/by-sa/4.0/)
%


%%%%%%%%%%%%%%%%%%%%%%%%%%%%%%%%%%%%%%
%     Configuration
%%%%%%%%%%%%%%%%%%%%%%%%%%%%%%%%%%%%%%
%%% Themes: Awesome-CV
\documentclass[]{awesome-cv}
\usepackage{textcomp}
%%% Override a directory location for fonts(default: 'fonts/')
\fontdir[fonts/]

%%% Configure a directory location for sections
\newcommand*{\sectiondir}{resume/}

%%% Override color
% Awesome Colors: awesome-emerald, awesome-skyblue, awesome-red, awesome-pink, awesome-orange
%                 awesome-nephritis, awesome-concrete, awesome-darknight
%% Color for highlight
% Define your custom color if you don't like awesome colors
\colorlet{awesome}{awesome-red}
%\definecolor{awesome}{HTML}{CA63A8}
%% Colors for text
%\definecolor{darktext}{HTML}{414141}
%\definecolor{text}{HTML}{414141}
%\definecolor{graytext}{HTML}{414141}
%\definecolor{lighttext}{HTML}{414141}

%%% Override a separator for social informations in header(default: ' | ')
%\headersocialsep[\quad\textbar\quad]
    \begin{document}
    
%%%%%%%%%%%%%%%%%%%%%%%%%%%%%%%%%%%%%%
%     Profile
%%%%%%%%%%%%%%%%%%%%%%%%%%%%%%%%%%%%%%
\begin{center}
	\headerfirstnamestyle{Daniel} \headerlastnamestyle{Hernandez} \\
	\vspace{2mm}
	{\hspace{0.8cm}\faEnvelope\ daniher02@gmail.com}  |  {\faMobile\ (+34) 744786798}  |  {\faMapMarker\ Málaga, España} 
	\newline {\faLink\ \href{https://www.linkedin.com/in/dhernandez-dev/}{https://www.linkedin.com/in/dhernandez-dev/}}
\end{center}

\cvsection{Habilidades}
\begin{cventries}
	\cventry
	{}
	{\def\arraystretch{1.15}{\begin{tabular}{ l l }
		Lenguajes:  & {\skill{ Python, R, SQL, Java, JavaScript, TypeScript, HTML, CSS}} \\
		Frameworks/Librerías:  & {\skill{ Django, FastAPI, Pandas, Apache Airflow, React, Spring Boot}} \\
		Bases de Datos:  & {\skill{ PostgreSQL, MySQL, Snowflake, Redis, SQLite}} \\
		Cloud y Herramientas: & {\skill{ AWS (Lambda, SNS, EC2, S3), Azure, Docker, Git, Metabase}} \\
		Herramientas de Datos: & {\skill{ Airflow, Astronomer, Kafka, Celery, ETL Pipelines, MCP Servers}} \\
		Metodologías:  & {\skill{ Agile, Microservicios, DDD, Event Driven Architecture, Data Engineering}} \\
		\end{tabular}}}
	{}
	{}
	{}
\end{cventries}
\vspace{-7mm}

%%%%%%%%%%%%%%%%%%%%%%%%%%%%%%%%%%%%%%
%     Experience
%%%%%%%%%%%%%%%%%%%%%%%%%%%%%%%%%%%%%%
\cvsection{Experiencia}
\begin{cventries}
	\cventry
	{Sports Data Engineer - Freelance}
	{Club Universidad de Chile - Departamento D+I}
	{Santiago, Chile}
	{Septiembre 2024 – Actualmente}
	{\begin{cvitems}
		\vspace{0.5mm}
		\item {Desarrollar pipelines de datos ETL para extracción automatizada de estadísticas de jugadores desde fuentes como Wyscout, integrándolos en bases de datos PostgreSQL para análisis avanzado y toma de decisiones deportivas.}
		\item {Implementar integración de Large Language Models en procesos de scouting mediante servidores MCP, automatizando análisis de jugadores y generación de reportes técnicos con IA.}
		\item {Desplegar y configurar servidor de Metabase en Microsoft Azure para migración de reportes desde PowerBI, estableciendo conexiones directas a bases de datos propias y mejorando visualización de datos.}
		\item {Construir plataforma web de centralización y visualización de datos departamentales con Django, desplegada en Azure usando Docker y Caprover para gestión de contenedores.}
		\end{cvitems}}

	\cventry
	{Desarrollador Full Stack}
	{Agencia de Marketing Digital IdeasPro}
	{Talca, Chile}
	{Marzo 2022 – Febrero 2025}
	{\begin{cvitems}
		\vspace{0.5mm}
		\item {Liderar equipo de 4 desarrolladores en proyectos de software para clientes, gestionando arquitectura técnica y entrega de soluciones escalables.}
		\item {Diseñar y desarrollar plataforma de gestión de datos administrativos y clínicos para empresas de radiografías dentales, optimizando procesos de atención en 50\% utilizando PHP y Java con Spring Boot.}
		\end{cvitems}}

\end{cventries}

\cvsection{Proyectos Destacados}
\begin{cventries}
	\vspace{-3mm}
	\cventry
	{}
	{Pipeline de Datos Ligas Europeas \vspace{-5mm}}
	{Apache Airflow, Snowflake, Astronomer, Python \vspace{-5mm}}
	{}
	{\begin{cvsectionnormaltext}
		\item {Pipeline ETL que automatiza extracción y carga de estadísticas de 7 ligas europeas. Procesamiento paralelo usando Task Groups en Airflow, reduciendo tiempo de 50+ minutos a ~12 minutos con carga a data warehouse Snowflake.
		\newline \faLink\ \href{https://github.com/Daniher01/european-football-pipeline}{https://github.com/Daniher01/european-football-pipeline}}
	\end{cvsectionnormaltext}}

	\vspace{-3mm}
	\cventry
	{}
	{Soccer Analytics MCP \vspace{-5mm}}
	{Python, MCP Server, Pandas, FastMCP \vspace{-5mm}}
	{}
	{\begin{cvsectionnormaltext}
		\item {Implementé el modelo estadístico DatAzul como servidor MCP para análisis y recomendación de jugadores de fútbol. Sistema que permite a LLMs como Claude analizar jugadores basándose en estadísticas objetivas con 8 posiciones configuradas.
		\newline \faLink\ \href{https://github.com/Daniher01/soccer-analytics-mcp}{https://github.com/Daniher01/soccer-analytics-mcp}}
	\end{cvsectionnormaltext}}

	\vspace{-3mm}
	\cventry
	{}
	{Weather Forecast System \vspace{-5mm}}
	{AWS Lambda, EventBridge, SNS, Python \vspace{-5mm}}
	{}
	{\begin{cvsectionnormaltext}
		\item {Sistema automatizado de alertas meteorológicas con arquitectura serverless. Pipeline de datos que consume APIs meteorológicas y distribuye notificaciones usando EventBridge y SNS, con reducción de costos del 80\% vs. infraestructura tradicional.
		\newline \faLink\ \href{https://github.com/Daniher01/weather-forecast}{https://github.com/Daniher01/weather-forecast}}
	\end{cvsectionnormaltext}}

	\vspace{-3mm}
	\cventry
	{}
	{Scouting de Jugadores \vspace{-5mm}}
	{R, Shiny App, StatsBomb \vspace{-5mm}}
	{}
	{\begin{cvsectionnormaltext}
		\item {Plataforma analítica con datos de StatsBomb para búsqueda de jugadores y análisis de similitudes. Implementación de algoritmos de clustering y visualización interactiva para scouting deportivo.
		\newline \faLink\ \href{https://dhernandezm.shinyapps.io/scouting_app/}{https://dhernandezm.shinyapps.io/scouting\_app/}}
	\end{cvsectionnormaltext}}

	\vspace{-3mm}
	\cventry
	{}
	{Sistema Gestión Datazul \vspace{-5mm}}
	{Django, Docker, PostgreSQL, Azure \vspace{-5mm}}
	{}
	{\begin{cvsectionnormaltext}
		\item {Plataforma web para centralización y visualización de datos deportivos. Sistema desplegado con Docker y base de datos PostgreSQL en Azure, incluyendo automatización de carga desde Excel y APIs externas.
		\newline \faLink\ \href{https://www.datazul.cl/}{https://www.datazul.cl/}}
	\end{cvsectionnormaltext}}
	
	\vspace{-5mm}
	
\end{cventries}

%%%%%%%%%%%%%%%%%%%%%%%%%%%%%%%%%%%%%%
%     Education
%%%%%%%%%%%%%%%%%%%%%%%%%%%%%%%%%%%%%%
\vspace{4mm}
\cvsection{Educación y Certificaciones}
\begin{cventries}
	\vspace{-3mm}
	\cventry
	{}
	{La Pizarra del DT \vspace{-5mm}}
	{Certificación Internacional \vspace{-5mm}}
	{2023 \vspace{-5mm}}
	{\begin{cvsectionnormaltext} 
		\item{Curso Big Data en el Fútbol - Análisis Avanzado de Datos Deportivos}
	\end{cvsectionnormaltext}}

	\vspace{-3mm}
	\cventry
	{}
	{La Pizarra del DT \vspace{-5mm}}
	{Certificación Internacional \vspace{-5mm}}
	{2023 \vspace{-5mm}}
	{\begin{cvsectionnormaltext} 
		\item{Curso Visualización 2D en el Fútbol - Representación Gráfica de Datos}
	\end{cvsectionnormaltext}}

	\vspace{-3mm}
	\cventry
	{}
	{CFT San Agustín \vspace{-5mm}}
	{Talca, Chile \vspace{-5mm}}
	{2020 - 2022 \vspace{-5mm}}
	{\begin{cvsectionnormaltext} 
		\item{Técnico en Análisis y Programación de Sistemas}
	\end{cvsectionnormaltext}}
\end{cventries}

\end{document}